%!TEX TS-options = --shell-escape
%!TEX TS-program = pdflatex
\documentclass[%
   10pt,              % Schriftgroesse
                 % wird an andere Pakete weitergereicht
   a4paper,           % Seitengroesse
   DIV10,             % Textbereichsgroesse (siehe Koma Skript Dokumentation !)
]{scrartcl}%     Klassen: scrartcl, scrreprt, scrbook, article
% -------------------------------------------------------------------------

\usepackage[T2A]{fontenc}

\usepackage[utf8]{inputenc} % Font Encoding, benoetigt fuer Umlaute
\usepackage[ngerman]{babel}   % Spracheinstellung


\usepackage{ulem}
\usepackage{graphicx}
\usepackage{amsfonts}
\usepackage{amsmath}
\usepackage{hyperref}
\usepackage{enumitem}
\usepackage{tikz}
\usepackage{multirow}
\usepackage{listings}
\usepackage{ifthen}
\usepackage{todonotes}
\usetikzlibrary{automata,arrows}
\usepackage{pgfplots}
\usepackage{booktabs}



% Definition des Headers
\usepackage{geometry}
\geometry{a4paper, top=3cm, left=3cm, right=3cm, bottom=3cm, headsep=0mm, footskip=0mm}
\renewcommand{\baselinestretch}{1.3}\normalsize
%\renewcommand{\labelenumi}{\alph{enumi})}
\newcommand{\norm}[1]{\left\lVert#1\right\rVert}
\def\header#1#2#3#4#5#6{\pagestyle{empty}
\noindent
\begin{minipage}[t]{0.6\textwidth}
\begin{flushleft}
\textbf{#4}\\% Fach
#6\\% Semester
#2  % Tutor 
\end{flushleft}
\end{minipage}
\begin{minipage}[t]{0.4\textwidth}
\begin{flushright}
\vspace*{0.2cm}
#5%  Names
\end{flushright}
\end{minipage}

\begin{center}
{\Large\textbf{ #1}} % Blatt

{(#3)} % Abgabedatum
\end{center}
}

\newenvironment{vartab}[1]
{
    \begin{tabular}{ |c@{} *{#1}{c|} } %\hline
}{
    \end{tabular}
}

\newcommand{\myformat}[1]{& #1}

\newcommand{\entry}[1]{
  \edef\result{\csvloop[\myformat]{#1}}
  \result \\ \hline
}

\newcommand{\numbers}[1]{
  \newcounter{ctra}
\setcounter{ctra}{1}
\whiledo {\value{ctra} < #1}%
{%
  \myformat{\thectra}
  \stepcounter{ctra}%
}
\myformat{\thectra}
}
\newcommand{\emptyLine}[1]{
  \newcounter{ctra1}
\setcounter{ctra}{1}
\whiledo {\value{ctra1} < #1}%
{%
  \myformat{\hspace*{0.5cm}}
  \stepcounter{ctra1}%
}
}

\begin{document}
%\header{Blatt}{Tutor}{Abgabedatum}{Vorlesung}{Bearbeiter}{Semester}{Anzahl Aufgaben}
\header{Skin Cancer Detection}{Prof. Schilling}{20. Dezember 2017}{Praktikum Maschinelles Lernen}{Florence Lopez \\ Jonas Einig \\ Julian Späth}{WS 17/18}

\textbf{TODOS für die Weihnachtsferien }
\begin{itemize}
	\item Fix tensorflow dataset api wrapper ($\rightarrow$ richtig sortieren) 
	\item Evaluationsskript/Testskript: Schwierigkeit ist richtiges Laden des alten (fertig trainierten) Netzes, da Augmentierung nicht im Graphen selbst enthalten sein darf. 
	\item richtiges Laden der Daten mit Dataset-Files (training, test, validation). Am besten eine Liste von Tupeln mit Labels und Images, gucken, dass beide dann gleich lang sind (Kombination dann über Python-Befehl zip).
	\item simple Android-App
	\item Cluster zum Laufen bringen (evtl. in C214 auf den Rechnern)
	\item Batch-Size (mindestens 2, besser wären 4) implementieren (wenn alles auf den Clustern läuft) für Simple Data Loader und Dataset api Wrapper 
	\item Zwischenpräsentation erstellen (GoogleDocs) 
\end{itemize}


\end{document}