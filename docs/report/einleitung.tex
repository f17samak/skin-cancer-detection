\section{Einleitung}

Hautkrebs gilt als eine der häufigsten Krebserkrankungen der Welt. Jährlich erkranken etwa 18.000 Menschen in Deutschland an dieser Krankheit, wobei Hautkrebs allgemein etwa für ein Prozent der Krebstodesfälle verantwortlich ist \citep{hautkrebs}. Findet eine Erkennung der malignen Hautläsionen frühzeitig statt, so ist es in den meisten Fällen möglich einen tödlichen Verlauf der Krankheit zu verhindern. Daher sind frühzeitige Erkennungssysteme sehr wichtig für die Bekämpfung von Hautkrebs.

Eine Ergänzung zum regelmäßigen Arztbesuch und dem damit verbundenem Hautscreening können daher neuronale Netze bieten, die aufgrund von medizinischen Datenbanken lernen zwischen malignen und benignen Hautläsion zu unterscheiden. Die genaue Implementierung und das Training eines solchen neuronalen Netzes werden wir dieser Arbeit genauer erläutern. 