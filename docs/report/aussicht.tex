\section{Aussicht und Diskussion}

Wie unsere Ergebnisse zeigen konnten, ist es also durchaus möglich einen Klassifikator zu erstellen, der Bilder von Hautläsionen in maligne und benigne Läsionen unterteilt. Dieser Klassifikator kann nun in verschiedenen Bereiche erfolgreich eingesetzt werden. Einen der Anwendungsbereiche betrachteten wir in unserem Projekt noch etwas genauer: Wir wollten den erstellten Klassifikator in eine mobile Anwendung einbinden, die es Nutzern ermöglicht vorerst ohne ärztlichen Rat ihre Haut auf Unregelmäßigkeiten und Anomalien zu untersuchen und eine vage Einschätzung dieser vorzunehmen. Dazu programmierten wir mittels Android Studio eine mobile Applikation, bei der ein Nutzer mittels seiner Handykamera ein Foto eines Muttermals oder einer Läsion aufnehmen kann und diese dann durch unseren Klassifikator ausgewertet wird. 

Natürlich ist die Leistung, die eine solche mobile Anwendung bietet, ohne jegliche Gewähr, da eine Handykamera aufgrund ihrer fehlenden Auflösung durchaus nicht einen ausgebildeten Hautarzt ersetzen kann. Trotzdem ist es durch diese App möglich eine erste Einschätzung des Hautzustandes vorzunehmen. Da wir weiterhin bei der Erstellung des Klassifikators sehr großen Wert darauf gelegt haben, vor allem die Rate der falsch Negativen zu minimieren, rät die mobile Anwendung dem Nutzer öfters dazu einen Hautarzt aufzusuchen, als es vielleicht wirklich nötig wäre. 


Zusätzlich könnte eine solche Handy-Applikation auch in Kombination mit einem speziellen Kamera-Aufsatz für präzise Aufnahmen der Haut genutzt werden. Damit wäre es möglich, die Genauigkeit der Klassifikation zu erhöhen und dadurch bessere Aussagen über den Gesundheitszustand des Nutzers zu treffen. Durch diese neue und einfache Art der Hautuntersuchung, wäre es vor allem auch Nutzern in ländlicheren Regionen, in denen ein Fachärzte-Mangel herrscht, möglich, eventuelle Unsicherheiten und Fragen bezüglich ihres Hautzustandes zu überprüfen. Obendrein birgt der Gebrauch von maschinellem Lernen in der Medizin in Kombination mit einer mobilen Anwendung einen weiteren großen Vorteil: die Bevölkerung erhält dadurch die Gelegenheit sich von zu Hause aus mit wichtigen medizinischen Themen und Untersuchungen zu beschäftigen. Somit wird das Bewusstsein für schwere Krankheiten, wie zum Beispiel Hautkrebs, und die regelmäßige Auseinandersetzung mit ihnen gefördert. Infolgedessen wäre es möglich eine Prophylaxe vom eigenen Zuhause aus durchzuführen, was wiederum in einer frühzeitigen Erkennung von Anomalien und damit einhergehend in einer schnellen und effizienten Behandlung dieser resultiert.\newline

In unserem Projekt beschäftigten wir uns vorerst mit der allgemeinen Unterscheidung zwischen benignen und malignen Hautläsionen. Damit unterschied sich unsere Problemstellung von der des Originalpapers von \citet{esteva2017dermatologist} erheblich und wurde deutlich vereinfacht. Es wäre also in Zukunft möglich, die originale Fragestellung zu übernehmen und nicht nur zwischen benignen und malignen Hautläsionen, sondern auch zwischen den verschiedenen Unterklassen der Malignen zu unterscheiden. Damit wäre es möglich, durch die mobile App eine genauere Einschätzung am Patienten vorzunehmen und die jeweils benötigte Medikation besser auf die klassifizierte Läsion einzustellen.\newline

Abschließend konnten wir durch unser Projekt zeigen, dass es durch maschinelles Lernen möglich ist, bösartige Hautveränderungen frühzeitig als diese zu erkennen. Diese Art der Klassifizierung bringt in der Medizin viele Vorteile mit sich, nicht zuletzt, da es dem Patienten dadurch möglich wird eine erste eigene Einschätzung des Gesundheitszustandes vorzunehmen. Natürlich kann die Richtigkeit einer solchen Klassifizierung nicht in dem Maße garantiert werden, in dem sie durch einen fachlich ausgebildeten Hautarzt gegeben ist, dennoch kann sie einen ersten Anhaltspunkt liefern und dazu führen, dass sich Patienten intensiver mit der Krankheit und ihrer Gesundheit auseinandersetzen. Vor allem durch die Einbindung des Klassifikators in eine benutzerfreundlichen mobile Applikation, kann der Vorteil des maschinellen Lernens in diesem Fall maximal ausgenutzt werden.  
 

\color{red}
\begin{itemize}
	\item[\checkmark] Einbindung in App $\rightarrow$ unsere Fortschritte dort beschreiben
	\item[\checkmark] Verknüpfung des Klassifikators mit speziellen Kameras fürs Handy
	\item[\checkmark] Vorteil von ML in Medizin: Bewusstsein für Skin Cancer kann erhöht werden, einfachere Methode grad in ländlichen Gegenden (wo es nicht viele Ärzte gibt)
	\item[\checkmark] genauere Unterscheidung wie auch im Paper noch möglich (dass einzelne Subtypen auch erkannt und unterschieden werden)
	\item[\checkmark] Abschluss-Absatz

\end{itemize}
\color{black}