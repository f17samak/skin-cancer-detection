\section{Problemstellung und Zielsetzung}

Im Rahmen des Praktikums ``Maschinelles Lernen'' beschäftigen wir uns mit der folgenden Problemstellung: Ist es möglich einen Klassifikator zu entwickeln, der Bilder von Hautläsionen in maligne und benigne Läsionen unterteilen kann. Maligne Hautläsionen sind die Läsionen, die für den Menschen gefährlich bis sogar tödlich verlaufen können, während benigne Hautläsionen gutartig und ungefährlich sind. Unser Projekt basiert dabei auf der Arbeit von \citet{esteva2017dermatologist}, wobei wir die originale Problemstellung jedoch etwas abgewandelt haben. Während \citet{esteva2017dermatologist} viele verschiedene Arten von Hautläsionen unterschieden haben, unterscheiden wir lediglich binär, zwischen zwei Klassen, nämlich den gutartigen und den bösartigen Läsionen. Dies vereinfacht die Problemstellung.

Unser Ziel war es, eine möglichst hohe Genauigkeit zu erreichen und vor allem die Anzahl der falsch negativen Vorhersagen möglichst gering zu halten. Im Zweifel soll der Klassifikator eher eine Läsion als maligne klassifizieren, auch wenn sie eigentlich benigne ist, anstatt eine maligne Läsion, die tödlich verlaufen könnte, zu verharmlosen und als benigne zu klassifizieren. Im Folgenden werden wir genauer auf die Methodik eingehen, die hinter unserem Klassifikator steckt und welche Ergebnisse dieser auf unbekannten Bildern von Hautläsionen liefert. 