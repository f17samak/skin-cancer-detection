\section{Ergebnisse}
\color{red}
\begin{itemize}
	\item besten Trainingsdurchlauf genau beschreiben $\rightarrow$ Lernrate, Loss, ...
	\item Evaluierungswerte aus diesem Durchlauf nennen und beschreiben bzw. erklären
	\item viele Bildchen :-) 
	\item Bilder aus Tensorboard
	\item Matlab-Plots und Tabellen 
    \item Threshold, Table, Plots von bestem durchgang
\end{itemize}
\color{black}

Um einen akzeptablen Klassfizierer zu bekommen waren einige Trainingsdurchläufe notwendig. Die anfänglichen Klassifizierer haben meist nur minimal bessere Ergebnisse als der Zufall geliefert. Gravierende Verbesserungen waren direkt sichtbar nachdem wir den Datensatz gemischt haben. Dies war notwendig, da die Daten geordnet vorlagen. Außerdem verringerten wir die Lernrate.

Einige ausgewählte Ergebnisse werden in Abbildung~\ref{fig:roc} als ROC-Kurve gezeigt.




