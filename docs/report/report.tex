\documentclass[a4paper, doc]{apa6}
\usepackage[utf8]{inputenc}
\usepackage[ngerman]{babel}
\usepackage[T1]{fontenc}
\usepackage{setspace}
\onehalfspacing
\usepackage{amsmath}
\usepackage{amsfonts}
\usepackage{amssymb}
\usepackage{graphicx}
\usepackage{listings}
\usepackage{calc}
\usepackage{natbib}
\usepackage{lipsum}



\title{Projektbericht: Skin Cancer Detection}
\shorttitle{Skin Cancer Detection}
\author{Florence Lopez, Jonas Einig, Julian Späth }
\affiliation{Department of Computer Science, University of Tübingen}


\begin{document}

\maketitle
\begin{abstract}
    
    \lipsum[1]
\end{abstract}
    
    
\section{Einleitung}

Hautkrebs gilt als eine der häufigsten Krebserkrankungen der Welt. Jährlich erkranken etwa 18.000 Menschen in Deutschland an dieser Krankheit, wobei Hautkrebs allgemein etwa für ein Prozent der Krebstodesfälle verantwortlich ist \citet{hautkrebs}. Findet eine Erkennung der malignen Hautläsionen frühzeitig statt, so ist es in den meisten Fällen möglich einen tödlichen Verlauf der Krankheit zu verhindern. Daher sind frühzeitige Erkennungssysteme sehr wichtig für die Bekämpfung von Hautkrebs.\\
\noindent Eine Ergänzung zum regelmäßigen Arztbesuch und dem damit verbundenem Hautscreening, können daher neuronale Netze bieten, die aufgrund von medizinischen Datenbanken lernen können, eine maligne von einer benignen Hautläsion zu unterscheiden. Die genaue Implementierung und das Training solcher neuronalen Netzen werden wir im Folgenden genauer erklären. 

\section{Problemstellung und Zielsetzung}

Im Rahmen des Machine Learning Praktikums beschäftigen wir uns mit der folgenden Problemstellung: unser Ziel ist es, einen Klassifikator zu entwickeln, der Bilder von Hautläsionen in maligne und benigne Läsionen unterteilen kann. Maligne Hautläsionen sind die Läsionen, die für den Menschen gefährlich bis sogar tödlich werden könne, während benigne Hautläsionen die gutartigen Läsionen darstellen. Unser Projekt basiert dabei auf der Arbeit von \citet{esteva2017dermatologist}, wobei wir die originale Problemstellung jedoch etwas abgewandelt haben. Während \citet{esteva2017dermatologist} viele verschiedene Arten von Hautläsionen unterschieden haben, wollen wir lediglich zwischen zwei Klassen, nämlich den gutartigen und den bösartigen Läsionen, unterscheiden, was die Problemstellung etwas vereinfacht.\\
Unser Ziel war es eine möglichst hohe Genauigkeit zu erreichen und vor allem die Anzahl der falsch Negativen möglichst gering zu halten. Denn im Zweifel soll der Klassifikator eine Läsion als maligne klassifizieren, auch wenn sie eigentlich benigne ist, anstatt eine maligne Läsion, die tödlich verlaufen könnte, zu verharmlosen und als benigne zu klassifizieren. Im Folgenden werden wir genauer auf die Methodik eingehen, die hinter unserem Klassifikator steckt und welche Ergebnisse dieser auf unbekannte Bilder von Hautläsionen liefert. 

\section{Methoden und Tools}

\begin{itemize}
	\item Python, Tensorflow, Scikit Learn, NumPy
	\item Skalierung der Bilder
	\item Augmentierungsmethoden
	\item Trainingsparameter (Anzahl Durchläufe, Lernrate, Loss-Funktionen)
	\item Evaluationsmethoden: Berechnung des Scores, etc.
	\item Aufteilung des Datensatzes in Training, Test, Validierung
	\item genaue Erklärung des Shufflings, da Datensatz nicht ausbalanciert
\end{itemize}

\section{Ergebnisse}
\begin{itemize}
	\item besten Trainingsdurchlauf genau beschreiben $\rightarrow$ Lernrate, Loss, ...
	\item Evaluierungswerte aus diesem Durchlauf nennen und beschreiben bzw. erklären
	\item viele Bildchen :-) 
	\item Bilder aus Tensorboard
	\item Matlab-Plots und Tabellen 
\end{itemize}

\subsection{Analysemethoden}

\begin{itemize}
	\item MCC, F1, F2
	\item threshold-Bestimmung 
\end{itemize}

\subsection{MCC}

\subsection{Genauigkeit}

\section{Aussicht}
\subsection{Einbindung in mobile App}

\begin{itemize}
	\item Einbindung in App $\rightarrow$ unsere Fortschritte dort beschreiben
	\item Verknüpfung des Klassifikators mit speziellen Kameras fürs Handy
	\item Vorteil von ML in Medizin: Bewusstsein für Skin Cancer kann erhöht werden, einfachere Methode grad in ländlichen Gegenden (wo es nicht viele Ärzte gibt)
	\item genauere Unterscheidung wie auch im Paper noch möglich (dass einzelne Subtypen auch erkannt und unterschieden werden)
\end{itemize}


\bibliography{mylit}
\bibliographystyle{apalike}

\end{document}