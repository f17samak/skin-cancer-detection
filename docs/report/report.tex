\documentclass[12pt,a4paper]{article}
\usepackage[utf8]{inputenc}
\usepackage[english]{babel}
\usepackage[T1]{fontenc}
\usepackage{setspace}
\onehalfspacing
\usepackage{amsmath}
\usepackage{amsfonts}
\usepackage{amssymb}
\usepackage{graphicx}
\usepackage{listings}
\usepackage{calc}
\usepackage{natbib}

\begin{document}
	
\begin{titlepage}
\begin{flushleft}
\begin{tabular}{p{\widthof{florence.lopez@student.uni-tuebingen.de}}p{13cm}}
				Julian Späth, 3938726&\\
				Jonas Einig&\\
				Florence Lopez, 3878792
\end{tabular}
\end{flushleft}
		\vspace*{3cm}
\begin{center}
			{\Large Skin Cancer Detection\bigskip\bigskip\\}
			{\large Eine Ausarbeitung zum \glqq Machine Learning Praktikum\grqq~bei Prof. Schilling im Wintersemester 2017/18}
\end{center}
		\hfill
\end{titlepage}
	
\titlepage

\section{Einleitung}

Hautkrebs gilt als eine der häufigsten Krebserkrankungen der Welt. Jährlich erkranken etwa 18.000 Menschen in Deutschland an dieser Krankheit, wobei Hautkrebs allgemein etwa für ein Prozent der Krebstodesfälle verantworlich ist \citet{hautkrebs}. Findet eine Erkennung der malignen Hautläsionen frühzeitig statt, so ist es in den meisten Fällen möglich einen tödlichen Verlauf der Krankheit zu verhindern. Daher sind frühzeitige Erkennungssysteme sehr wichtig für die Bekämpfung von Hautkrebs.\\
\noindent Eine Ergänzung zum regelmäßigen Arztbesuch und dem damit verbundenem Hautscreening, können daher neuronale Netze bieten, die aufgrund von medizinischen Datenbanken lernen können, eine maligne von einer benignen Hautläsion zu unterscheiden. Die genaue Implementierung und das Training solcher neuronalen Netzen werden wir im Folgenden genauer erklären. 

\section{Problemstellung und Zielsetzung}

Im Rahmen des Machine Learning Praktikums beschäftigen wir uns mit der folgenden Problemstellung: unser Ziel ist es, einen Klassifikator zu entwickeln, der Bilder von Hautläsionen in maligne und benigne Läsionen unterteilen kann. Maligne Hautläsionen sind die Läsionen, die für den Menschen gefährlich bis sogar tödlich werden könne, während benigne Hautläsionen die gutartigen Läsionen darstellen. Unser Projekt basiert dabei auf der Arbeit von \citet{esteva2017dermatologist}, wobei wir die originale Problemstellung jedoch etwas abgewandelt haben. Während \citet{esteva2017dermatologist} viele verschiedene Arten von Hautläsionen unterscheiden wollten, wollen wir lediglich zwischen zwei Klassen unterscheiden, was die Problemstellung etwas vereinfacht.\\
\noindent 

\section{Methoden und Tools}

\section{Ergebnisse}

\subsection{ROC und AUC}

\subsection{MCC}

\subsection{Genauigkeit}

\section{Aussicht}


\bibliography{mylit}
\bibliographystyle{unsrt}

\end{document}